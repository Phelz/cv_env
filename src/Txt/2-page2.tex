\section{Experience}

\begin{tblr}{%
    colspec={ t{5.25cm}  X[l,8] X[l,5.5]  } ,
    column{1} = {font=\bfseries\fontsize{11}{12}\selectfont},  % 12pt font with 14pt line spacing
    column{2} = {font=\fontsize{11}{11}\selectfont},           % 11pt font with 13pt line spacing
    column{3} = {font=\fontsize{11}{11}\selectfont},           % Same for column 3
    width=1.0\textwidth,
    colsep = 0.1pt, % Adjust column spacing for this table
    rowsep=0.001\baselineskip} % Adjust row spacing
    % Table contents go here
    Teaching Assistant (TA): & Department of Physics and Astronomy & \CVlocation{University of Calgary} \\
\end{tblr}

\begin{ta_table}

	Lab TA   &  PHYS 211: Mechanics & \CVdate{Fall, 2024}   \\
	Lead TA  &  PHYS 355: ELectromagnetic Theory & \CVdate{Winter, 2024}   \\
	Lab TA   &  PHYS 397: Applied Physics Labratory & \CVdate{Fall, 2023}   \\
	Lab TA   &  PHYS 229: Modern Physics & \CVdate{Winter, 2021}   \\
	Lead TA  &  PHYS 259: Electricity and Magnetism & \CVdate{Summer, 2020}   \\
	Lab TA   &  PHYS 259: Electricity and Magnetism & \CVdate{Winter, 2020}   \\

\end{ta_table}

\begin{tblr}{%
    colspec={ t{5.25cm}  X[l,6] X[l,5]  } ,
    column{1} = {font=\bfseries\fontsize{11}{12}\selectfont},  % 12pt font with 14pt line spacing
    column{2} = {font=\fontsize{10.5}{12}\selectfont},           % 11pt font with 13pt line spacing
    column{3} = {font=\fontsize{10.5}{12}\selectfont},           % Same for column 3
    width=1.0\textwidth,
    colsep = 1pt, % Adjust column spacing for this table
    rowsep=0.001\baselineskip} % Adjust row spacing
    % Table contents go here
    Learning Assistant (LA): & Department of Computer Science & \CVlocation{University of Calgary} \\
\end{tblr}

\begin{ta_table}

	Tutorial LA  &  DATA 202: Thinking with Data & \CVdate{Winter, 2021}   \\
	Tutorial LA  &  CPSC 501: Advanced Programming Techniques & \CVdate{Fall, 2020}   \\

\end{ta_table}


\begin{tblr}{%
    colspec={ t{5.25cm}  X[l,6] X[l,7.5]  } ,
    column{1} = {font=\bfseries\fontsize{11}{12}\selectfont},  % 12pt font with 14pt line spacing
    column{2} = {font=\fontsize{11}{11}\selectfont},           % 11pt font with 13pt line spacing
    column{3} = {font=\fontsize{11}{11}\selectfont},           % Same for column 3
    width=1.0\textwidth,
    colsep = 0.1pt, % Adjust column spacing for this table
    rowsep=0.001\baselineskip} % Adjust row spacing
    % Table contents go here
    Research Assistant: & Auroral Imaging Group & \CVevent{May, 2021 - January, 2023}{Calgary, AB, Canada} \\

\end{tblr}



\textbf{Research on High Frequency Radiowave Propagation:}
\begin{itemize}
    \item Developed a new technique for modeling radio-wave absorption
    associated with geomagnetic activity (e.g., solar storms) using a prototype instrumentation. The technique was
    published in: \textit{Journal of Geophysical Research: Space Physics}--\href{https://onlinelibrary.wiley.com/doi/abs/10.1029/2023JA032375}{Ghaly et al., 2024}
    \item Coordinated with Natural Resources Canada (NRCan) to incorporate their network of radio transmitters into my
    analysis, thereby developing a space weather forecasting/alerting system for modeling where and when radio
    communications are disrupted.
    \item Developed code that extends my methodology to a network of multiple radio transmitters and receivers, and allows
    for simulating the impact of geomagnetic events before they occur.
\end{itemize}

\textbf{Technical Experience, Data Generation, and Software Development:}
\begin{itemize}
    \item Spearheaded the realization of a prototype, Software-Defined Radio (SDR) to make it usable for research.
    \item Wrote software for data acquisition in Python and Java and tested the instrument's ability to function as a receiver.
    \item Developed an online data browser for the prototype instruments deployed to generate curated, raw (K0-level) data for the public. Back-processed and published K0 data for 2022 and 2023.
    \item Wrote instrument-side code to generate curated K0 data on-site for all instruments deployed--now being published in real-time.
    \item Developed an instrument baselining Graphical User Interface (GUI).
    \item Developed a methodology for real-time, automatic baselining to generate K2 (operational) data.
    \item Designed a web application with a Postgre database that browses through different specification parameters for flagging space weather events.

\end{itemize}




% \begin{job_title}
%     Research Assistant: & Auroral Imaging Group &  \CVdate{May, 2021 - January, 2023} & \CVlocation{Calgary, AB, Canada} \\
% \end{job_title}


% INFORMATION ABOUT PUBTITLE COMMAND:
%-------------------------------
% Produces a symbol with icon color and \large and bold text.
% To have a custom symbol and text, use \pubtitle[<text>][<symbol>]{custom}
% \pubtitle{custom} have default values, and you can use \pubtitle[<text>]{custom}
% to keep the default symbol and change the text or \pubtitle[<symbol>]{custom} to
% change the symbol and keep the default text (where symbol is a single token/
% braced group, i.e. a command like \faBeer). 
%
% Supported entry types:
% - custom
% - article
% - book
% - thesis
% - report
% - manual 
% - online
% - software
% - datatype
% - patent
% - conferance
% - inproceedings
% - masterthesis
% - phdthesis


%\end{CVbody}